\chapter{Related Work}\label{ch:RelatedWork}


\section{PDE-based image compression}

In recent years, a new image codec based on partial differential equations was developed as an
alternative to JPEG and JPEG2000. The core idea is to compress the image by selecting a set of
pixels that serves as the basis for an inpainting process. This inpainting process is defined by a
partial differential equation that is solved iteratively during the reconstruction phase.\\
In 2005, Galić et al.\cite{galic05}\ first introduced an alternative image compression method using nonlinear
anisotropic diffusion
 as
a serious alternative to more classical approaches like JPEG. In this
work, the authors showed the inpainting capabilities of a diffusion process known as edge-enhancing
diffusion, which has since then established itself as the prime choice for PDE-based image
compression. The approach they presented relies on efficiently storing a pixel mask that is later
on used to reconstruct the image by filling in the missing regions using the aforementioned process
of edge-enhancing diffusion and lays the groundwork for later publications building on the codec
defined herein. \\
As already mentioned, the codec relies on computing a pixel mask from the original image. However,
this process is a balancing act. On one hand, one wants to get a mask that yields a perfect
reconstruction of the original image but on the other hand, one also wants to create a mask that
can be encoded and stored efficiently, i.e. using the smallest amount of \textit{bits-per-pixel
(bpp)} possible. This step of mask computation is still topic of ongoing research which proves the
difficulty of this problem.\\
In the initial codec, also called the \textbf{BTTC-EED} codec, the mask was
computed by means of an adaptive sparsification scheme relying on B-tree triangular coding (BTTC)
that was already introduced by Distasi et al\cite{distasi97} back in 1997. This fairly simple
approach iteratively subdivides the image diagonally if the error of the reconstruction of the image using only
the corner points of the subdivision exceeds an a priori defined threshold. In this version, the
reconstruction is approximated by a simple linear interpolation inside the triangle. The efficiency
of constructing a mask using BTTC lies in the binary tree structure of the subdivision that can be encoded extremely
efficiently by a simple binary string. Furthermore, grey values are encoded using a straight
forward entropy
coding method like Huffman coding\cite{huffman}.
With this fairly simple approach they were already able to outperform JPEG visually for high
compression rates and comic-style images~\cite{galic05}.\\
In 2008, the same authors improved this version by adding a requantisation step to the encoding
phase, in which they quantised the grey values form 256 values to 64 in order to further shorten
the sequences obtained by Huffman coding.

%Improving on this, the authors published a new paper in 2008, adding a number of additional
%procedures to the compression phase, with which they were finally able to come close to the quality
%of JPEG 2000~\cite{galic08}.
%Finally, in 2009, Schmaltz et al.\ optimised the ideas even further, building the
%so called \textbf{R-EED} codec with which they could even beat JPEG 2000~\cite{schmaltz09}.
%The main differences between~\cite{galic05} and~\cite{schmaltz09} are the addition of several
%procedures to optimise the data set that is kept for inpainting in the decompression step.
%To roughly summarise the whole compression phase~\cite{schmaltz09}:\\
%First, an initial set of points is gathered by using a rectangular subdivision (instead of the
%previous triangular subdivision) of the image. This works by recursively splitting the image in
%half whenever the reconstruction using only the boundary points exceeds a certain error threshold.
%The reconstruction is also done using EED inpainting.
%After obtaining the initial data set, the brightness values of each of the kept pixels is rescaled
%to $[0, 255]$ to eliminate possible quantisation artifacts. Due to this brightness rescaling, the
%optimal contrast parameter for the decompression phase may change and thus has to be adjusted as
%well. This is generally done alternating between optimising points and the contrast
%parameter until a certain convergence criterion is met.
%Abs a last step, the authors invert the inpainting mask and perform the inpainting process to fill in
    %the kept data as a means to increase the coherence between the optimised pixels and the
    %original image.

    %All of these measures serve the purpose of decreasing the \textit{mean squared error (MSE)} to a
%level where the proposed codec is able to outperform JPEG 2000 for compression rates higher than
%$\mathbf{43:1}$.

\section{Image features in image compression}
Features such as edges and corners are very important in the field of image processing as they
provide almost all of the semantics of an image\cite{marr82}. Therefore, feature detection has been a staple in
this field for a long time. Actually, corner and edge detection algorithms such as the ones by John Canny
and Chris Harris\cite{harris88, canny86} are still used today. 
Despite their semantical importance, edges and corners have not had that much impact on image
compression so far.\\

In 2007, Zimmer introduced a new approach using corners to compute inpainting masks for
PDE-based image compression\cite{zimmer07}. In their approach, they used a simple corner detection method to
sample a set of the most important corners. The inpainting mask was then obtained by storing the
8-neighbourhood around each corner. In the reconstruction/inpainting phase they used a method
following the idea proposed by Bertalmio et al\cite{bertalmio00}. Instead of just using pure EED to
inpaint the image, they interleaved edge-enhancing diffusion with mean curvature motion in order to
improve the inpainting in regions with sparser inpainting domains. \\
Even though their codec was
not able to compete with widely used codecs like JPEG and JPEG2000 they proved that corners are
indeed useful for PDE based inpainting. They found the largest disadvantages to be the sparsity of
corners in images. The reasoning behind this is that because of the sparsity of corners in an
image, one has to either invest in a very sophisticated and complex inpainting mechanism to fill in
the large areas between the few detected corners \textit{or} adjust the corner detector to be more
`fuzzy'. On one hand, this leads to a denser mask and hence better results reconstruction-wise 
but on the other also creates suboptimal inpainting masks. Due to the fuzzy nature of the
corner detector, flat or homogeneous areas are classified as corners and therefore kept in the
mask, even though these regions could have easily been filled in by even a simple inpainting
process like linear diffusion. \\
To improve on their work, they suggested to touch on the parameter
selection for the corner detector, especially the selection of both the cornerness threshold and
integration scale used in the computation of the structure tensor since these heavily influence the
amount and quality of corners that are detected. As another improvement, they suggested different
shapes of corner regions. My thesis is largely built upon the results from this work and tries to
implement the improvement ideas proposed by Zimmer to see how well they work.\\

In \cite{mainberger09, mainberger10}, the authors implemented a diffusion based reconstruction
using mainly edges as the inpainting mask. For the mask construction they used the Marr-Hildreth
edge detector combined with \textit{hysteresis thresholding} as proposed by Canny\cite{canny86}.
They stated however, that they were not locked in on the Marr-Hildreth edge detector and that
others such as the classical Canny edge detector could also be used, especially for images that
contain a larger amount of blurry edges. The addition of hysteresis thresholding yielded closed and
well localised contours which they encoded using the lossless \textit{JBIG} encoding\cite{jbig}.
To reconstruct the image from the stored contours, they used a simple linear diffusion approach
which, in this case, was good enough. For cartoon-like images, they could beat the quality of
JPEG and even the more sophisticated JPEG2000 in terms of the PSNR (peak signal to noise ratio) of the reconstructed image to the
original one. This was all done in the earlier work\cite{mainberger09}. In the follow-up
publication from the same authors in 2010 they introduced some optimisations such as different
entropy coders for the edge locations as well as an optimised method of storing the grey/colour
values of the mask pixels. Another improvement was the use of an advanced method for solving the
linear systems arising from the inpainting problem. In contrast to the earlier publications a fast
\textit{full multigrid scheme} was implemented to speed up the decoding phase significantly to 
a point where the codec is now realtime capable.\\

In \cite{peter15}, region of interest(ROI) coding was first introduced for the R-EED codec. Previously,
it was not possible to specify certain regions of an image to be reconstructed with higher detail.
With this new contribution, they allowed the user to specify a weighting mask that is used during
the mask construction phase to adapt the mask in such a way that the specified regions would have a
denser mask than other regions. This is expecially useful in medical imaging, where some regions of
an image, e.g. in a CT scan of a brain region, need to be reconstructed exactly in order to prevent
false diagnoses. In this work however, the weight masks had to be chosen manually, which is not
always very practical (cf. section \ref{sec:Discussion}: Discussion).
\\
ROI coding was not the only contribution in this publication. The authors also proposed
\textit{progressive modes} for the R-EED codec to be able to compete better with widely available codecs.
Progressive modes allow an image to be displayed in a coarse manner, e.g. when shown on a website
that has not fully loaded the image data yet, and gradually refine the representation as more data
becomes available. Last but certainly not least, they showed that their PDE-based codec is also
capable of realtime video decoding and playback which sets a milestone on the way to prove the
\textit{suitability of R-EED for real-world applications}\cite{peter15}. In their words this
publication, or rather the extensions presented therein, mark the first step of an evolution of
PDE-based compression codecs from the proof-of-concept stage to fully grown codecs with relevance
for practical applications.\cite{peter15}\\

\section{Outline}
So far, we have reviewed some of the work that is closely related and motivated the goal of our work.
To explain what exactly was done in order to get to this goal, we first have to go over some of the
theoretical background. We will introduce basic concepts from
calculus and linear algebra (\ref{sec:Basics}) as well as some more advanced topics
like the structure tensor (\ref{sec:Structure}) and
inpainting methods (\ref{sec:Inpainting}).\\
In chapter \ref{ch:Implementation} we will then talk about the technical details of our 
implementation, for example the discretisation used to implement the theoretical ideas from 
the previous chapter (\ref{sec:Discretisation}). 
Furthermore, we will present some additions we made and the reasoning behind them
(\ref{sec:Contribution}).\\
Afterwards, we discuss the experiments that were concluded in order to evaluate the initial idea as
well as show the results these experiments yielded.\\
Last but not least in chapter \ref{ch:Conclusion}, we conclude our work, discuss the strengths
and shortcomings of our approach, h ow it could see some improvements and where it could be
applied to.\\

