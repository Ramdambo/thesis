\chapter{Conclusion and Outlook}\label{ch:Conclusion}

\section{What has been done?}

To summarise, I have adapted the method proposed in~\cite{zimmer07} by introducing circular corner
regions and explored their potential for PDE based inpainting and compression approaches.\\
I used a simple structure tensor based approach for corner detection, namely the Förstner-Harris
corner detector to find corner candidates. Using the new additional procedures called
\textit{circular non-maximum suppression} and \textit{total pixel percentile thresholding}, these
candidates were filtered to get a mask that satisfies a pre-defined pixel density criterion. With
these additions, aside from special cases, it was possible to construct masks of similar pixel
densities quite consistently. However, one should mention that both methods are pretty naive
approaches far from perfect.\\
I tested the hypothesis that the mask radius should be chosen in dependence of the integration
scale during the computation of the structure tensor. Moreover, the hypothesis stated that the mask
radius should always be larger than the integration scale in order to cover the possible inaccuracy
in the localisation of the corner. The experiments yielded reasonable evidence that this is the
case. 
All in all it was shown that by increasing the size of corner regions one can cover the uncertainty
introduced during the corner detection process and improve the reconstruction quality compared to
the method proposed in~\cite{zimmer07}.

\section{Discussion}

The method examined herein is pretty well suited for reconstruction/compression of binary images
with little to no texture. The problem in the compression of highly textured grey value images lies in the 
fact that due to the large corner regions, if one wants to cover highly textured regions, either
one chooses the corner regions fairly small to cover as much of the texture as possible, similar to
a subdivision based or stochastic approach. The downside here is that for small mask radii the
inaccuracy in the corner detection might influence the outcome. Another possibility is to choose
the corner regions larger such that more corners are covered by a single region. By doing this
however, the mask will become so dense that it will be suboptimal in a compression context.
Another point to mention, that also Zimmer's method\cite{zimmer07} struggled with, is the sparsity
of corners in images. While corners are semantically important features, they are pretty seldom in
organic images. (see the image \texttt{lena512} or \texttt{trui}).\\
Something to consider as well is the compression of noisy images. Here one has to pose the question
whether the decompressed version of these images should contain noise or not. On one hand, noise in
images is mostly a degradation that one wants to remove. On the other hand, image compression
mostly aims at reconstructing the images that were compressed. By applying PDE based inpainting
approaches in this context, most noise is removed due to the nature of these algorithms as we saw
in Section~\ref{sec:Results}, so the image that is decompressed is not the same as the one that was
compressed. While in most cases this could be considered a benefit of the procedure, one could
think of examples where such an inherent denoising would be considered unwanted behaviour (images
of white noise?).\\

\section{Future Work}\label{sec:FutureWork}

Lastly, we will discuss some possible improvements to corner region based inpainting/compression
approaches.

\subsection*{Improvement of the corner detection}

As we saw in Section~\ref{sec:CornerEx}, the accuracy of the corner localisation is key to a good
inpainting result. While the Förstner-Harris method is pretty accurate, one could think of other
methods that suggest higher, even sub-pixel accuracy. A possible method could be one proposed by
Alvarez et al.\cite{amss} in 1997. The method is based on a scale-space approach that is supposed
to be able to trace back the origin of a corner to its actual location with subpixel accuracy.

\subsection*{Optimising corner region radius}

Something that one could investigate further would be the optimal corner region radius and whether
it would be sensible to make it variable in space. The idea behind that is, that in regions where
corners are sparse, one could increase the radius to include more context from around the corner.
Similarly, in textured regions the radius could be decreased in order to get a mask that looks more
like a mask from a probabilistic sparsification or subdivision based approach.

\subsection*{Combination with ideas from other approaches}

Since it is pretty difficult to produce good inpainting masks, i.e. masks that result in
reconstructions with a high PSNR, only using corners and corner regions, one could consider
applying methods from other approaches to corner based mask creation. One method that comes into
mind would be probabilistic sparsification\cite{hoeltgen12, schmaltz14} applied to corner regions.\\
Seeing what impact tonal optimisation had\cite{hoeltgen12, schmaltz14}, one could also consider
applying this to a corner region based approach.

\subsection*{Possible application to region of interest(ROI) coding}

On a more unrelated note, corner regions could see application in region of interest coding as
proposed in~\cite{peter15}. Inpainting masks based on corner regions could be used as weighting
masks for the subdivision approach to be able to better reconstruct regions of higher semantic
importance as indicated by a higher density of corners.
