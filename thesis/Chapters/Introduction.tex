\chapter{Introduction}\label{ch:Intro}
As technology evolves, the quality and resolution of digital images improve as well. But as the
quality increases so does the memory required to store the image on a hard drive. To counteract
this increase in disk space usage, people have tried to reduce the sizes of digital images a lot in the
last decades.

\section{Motivation}
One of the most successful and probably most well known \textit{codecs}
is \textbf{JPEG}~\cite{wallace92} and its lesser known but more advanced successor \textbf{JPEG
2000}~\cite{jpeg2000}. Both are lossy image compression methods
known for fairly high compression rates while still providing a reasonable image quality.
For higher compression rates however, the quality deteriorates pretty quickly and the infamous ``block
artifacts'' are being introduced. As a remedy, a new class of image compression methods based on
\textit{digital image inpainting} have been developed over the last
years~\cite{galic05, galic08, mainberger12, zimmer07, mainberger09, mainberger10, dong07, schmaltz09} that
aim to create better looking images for higher compression rates than JPEG and even JPEG2000. \\
Normal use cases for inpainting methods include, among other things, restoring damaged paintings and removing 
objects from images but by pushing these methods to their limits, they can also be used for image compression.
This can be accomplished by carefully selecting a very small subset of all pixels in the image, 
and then use such an inpainting method to restore the image from only the given data.
As one can imagine, selecting the right data is a fairly minute process and one has to carefully
select the pixels to keep. Even though there has been a lot of work done in this
area~\cite{belhachmi09, schmaltz14, hoeltgen12}, the
selection can still be improved.\\
In the past, few methods explored how well corners or corner-like features perform as candidates for
the inpainting process. In~\cite{zimmer07}, a method for image compression using corners was
proposed, but was not able to come close to the performance of JPEG\@.
However, the full potential of this method was not explored. \newpage\noindent
One of the main points of criticism is,
that the corner localisation was not accurate enough, or rather the inaccuracies were not accounted 
for in the mask selection~\cite{conversation}. This is why I want to improve on this approach and try 
to tap into its unexplored potential.

\section{Goal}\label{sec:Goal}

The goal of this thesis is two-fold:
Firstly, we want to explore how inaccuracies in the corner localisation can be properly accounted
for in the mask creation. This includes finding out how large the corner region disks have to be to
counteract the inaccuracy introduced during the detection phase.\\
Secondly, we want to introduce some additional procedures to improve the selection of corner regions
and make it a bit more robust with respect to the pixel density in the final mask, as it is fairly
important to reliably produce masks containing only a certain percentage of pixels if one wants to
use this approach for image compression.
One of the biggest factors that we had to consider was that corners and similar features can be quite 
rare in images, severely limiting the amount of candidates one can choose from, especially compared to
approaches like~\cite{schmaltz09, hoeltgen12}. 
We will cover these procedures in Section~\ref{ch:Modelling} while the first goal will be covered
in Section~\ref{sec:ParameterSelection}.
