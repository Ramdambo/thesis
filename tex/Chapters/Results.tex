\chapter{Experiments} 

\label{ch:Experiments} 
\section{Test images}

\section{Parameter Selection}
In this section we will talk about how different parameter choices influence the results of the
respective algorithm and what we can do to make selection more intuitive. First, we will discuss
the different parameter choices for corner detection and how we can make it more intuitive. After
that, I will explain what parameters were ultimately chosen to achieve the best results.\\
In the second part, we will do the same thing for EED inpainting. Keep in mind that for most of the
parameters therein, we will use fixed values that were predetermined in previous works.
\subsection{Corner Detection}
For corner detection there are not that many different parameters to play around with. In the
classical approach using the structure tensor (cf section \ref{sub:Corner}) we have only 3
parameters:
\begin{itemize}
    \item the noise scale $\sigma$
    \item the integration scale $\rho$ and
    \item a threshold parameter $T$ to filter out non-important corners
\end{itemize}

Traditionally, one wants to choose the noise scale as large as necessary but keep it as small as
possible, meaning that we want to choose the smallest noise scale that gets rid of most of the
noise in the image. That is because with a larger $\sigma$ 
one often faces the problem that the detected corners can not be located as accurately anymore,
since more and more relevant features are smoothed away (cf. scale space section). Another problem
is that the gaussian scale space (iterated gaussian smoothing) may even introduce new corners.
 Most of the time however, a $\sigma$ of 1 is sufficient enough to remove most of the noise and unnecessary
 details and still provide an accurate result.\\
Next up is the integration scale $\rho$. The integration scale basically determines how `local' 
the corner detection is as it influences the size of the region structural information os averaged
in in the computation of the structure tensor. $\rho$ should always be chosen larger than the noise
scale $\sigma$, which leads us to a `standard' value of 2-2.5. In our experiments, these values
usually yielded the best results.\\
Now, in the classical version of the Harris corner detector which we used, the cornerness measure
is thresholded against some artificial threshold $T$. The problem with this approach is that the
value $T$ heavily depends on the input image and thus has to be chosen manually for every image. An
alternative approach that does not require to adapt this threshold was used in \cite{zimmer07}.
They fixed the threshold at a certain value and varied the integration scale based on the
compression rate they wanted to achieve.\\
In this work, I used another approach than the classical and the one by Zimmer\cite{zimmer07}.
I replaced the artificial threshold $T$ by a simple percentile parameter. 

\subsection{Inpainting}

\section{Results}\label{sec:Results}
